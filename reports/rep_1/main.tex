\documentclass[12pt]{report}
\usepackage[utf8]{inputenc}
\usepackage[spanish]{babel}
\usepackage{amsmath}
\usepackage{amssymb}
\usepackage{amsfonts}
\usepackage{mathrsfs}
\usepackage{geometry}
\usepackage{titlesec}
\usepackage{graphicx}
\usepackage{float}
\usepackage{comment}
\usepackage{listings}
\usepackage{hyperref}
\titlespacing*{\section}{0pt}{0.5ex}{1ex}
\titlespacing*{\subsection}{0pt}{0.5ex}{0.5ex}
\usepackage{xcolor}
\usepackage{pgfplots}
\usepackage{pgfplotstable}
\usepackage{subcaption}
\pgfplotsset{compat=1.18}
\usepackage{tikz}
\usetikzlibrary{babel}
\usepackage{palatino}
\usepackage{lettrine}
\usepackage{lipsum}
\usepackage{tikz-3dplot}

\hypersetup{
    colorlinks=true,
    linkcolor=blue,
    filecolor=magenta,      
    urlcolor=cyan,
    pdftitle={Overleaf Example},
    pdfpagemode=FullScreen,
}


\definecolor{bg}{RGB}{248,248,248}
\definecolor{frame}{RGB}{220,220,220}
\definecolor{comment}{RGB}{120,120,120}
\definecolor{keyword}{RGB}{30,90,160}
\definecolor{string}{RGB}{180,60,60}
\definecolor{linenumber}{RGB}{150,150,150}

\lstset{
  language=Python,
  basicstyle=\ttfamily\footnotesize,
  keywordstyle=\bfseries,
  commentstyle=\itshape,
  stringstyle=\ttfamily,
  tabsize=2
}

\begin{document}
\begin{titlepage}
\centering

\rule{\textwidth}{1pt}
{\LARGE\bfseries ANÁLISIS DE OPTIMIZACIÓN DE INVENTARIOS: CASA MONARCA \par}
\vspace{0.5cm}
{\large\scshape Optimización determinista\par}
\rule{\textwidth}{1pt}
\vspace{1cm}

{\large Angel Luna \par}
\vspace{0.3cm}
{\itshape Instituto Tecnológico y de Estudios Superiores de Monterrey\par}



\vspace{0.8cm}

\tdplotsetmaincoords{68}{110}

\begin{tikzpicture}[tdplot_main_coords, scale=3, line join=round, line cap=round]

% ---------- styles ----------
\tikzset{
  boxedge/.style={black!35, line width=0.35pt},
  boxdiag/.style={black!20, line width=0.25pt},
  axis/.style={black, ->, line width=0.55pt},
  strong/.style={black, line width=0.9pt},
  med/.style={black, line width=0.65pt},
  hidden/.style={black!45, dashed, line width=0.35pt},
  faceA/.style={fill=black!18, fill opacity=0.35},
  faceB/.style={fill=black!30, fill opacity=0.25},
  dot/.style={circle, fill=black, inner sep=0pt, minimum size=3.2pt},
}

% ---------- outer box ----------
\coordinate (A)  at (-1,-1,0);
\coordinate (B)  at ( 1,-1,0);
\coordinate (C)  at ( 1, 1,0);
\coordinate (D)  at (-1, 1,0);

\coordinate (Ap) at (-1,-1,1.2);
\coordinate (Bp) at ( 1,-1,1.2);
\coordinate (Cp) at ( 1, 1,1.2);
\coordinate (Dp) at (-1, 1,1.2);

% draw box (light)
\draw[boxedge] (A)--(B)--(C)--(D)--cycle;
\draw[boxedge] (Ap)--(Bp)--(Cp)--(Dp)--cycle;
\draw[boxedge] (A)--(Ap) (B)--(Bp) (C)--(Cp) (D)--(Dp);

% faint diagonals (optional but nice)
\draw[boxdiag] (A)--(Cp);
\draw[boxdiag] (D)--(Bp);

% ---------- axes + labels ----------
\coordinate (O)   at (0,0,0);
\coordinate (E12) at (-1.25,-0.75,0);
\coordinate (E23) at ( 1.35, 0.00,0);
\coordinate (E13) at ( 0.00, 0.00,1.45);

\draw[axis] (O)--(E12);
\draw[axis] (O)--(E23);
\draw[axis] (O)--(E13);

\node[below left] at (E12) {$\{1,2\}$};
\node[right]      at (E23) {$\{2,3\}$};
\node[above]      at (E13) {$\{1,3\}$};

% ---------- key outer triangle (bold) ----------
% choose the top triangle corners
\draw[strong] (Dp)--(Cp)--(Bp)--cycle;

% ---------- inner geometry (faces drawn back-to-front) ----------
% a "deep" apex and two nested hulls (tweak coords for exact match)
\coordinate (V) at (0.00, 0.00,-0.35);   % main bottom apex

% outer inner hull (like the big translucent wedge)
\coordinate (p1) at (-0.60, 0.65,0.95);
\coordinate (p2) at ( 0.85, 0.55,0.95);
\coordinate (p3) at ( 0.35,-0.60,0.88);

% inner hull (smaller wedge inside)
\coordinate (q1) at (-0.35, 0.42,0.78);
\coordinate (q2) at ( 0.55, 0.38,0.78);
\coordinate (q3) at ( 0.25,-0.35,0.66);

% Put fills in a transparency group so overlaps look smooth.
\begin{scope}[transparency group]

  % --- Draw "back" faces first (order matters!) ---
  \fill[faceB] (p1)--(p2)--(V)--cycle;
  \fill[faceB] (p2)--(p3)--(V)--cycle;
  \fill[faceB] (p3)--(p1)--(V)--cycle;

  \fill[faceA] (q1)--(q2)--(V)--cycle;
  \fill[faceA] (q2)--(q3)--(V)--cycle;
  \fill[faceA] (q3)--(q1)--(V)--cycle;

\end{scope}

% outlines (medium)
\draw[med] (p1)--(p2)--(p3)--cycle;
\draw[med] (p1)--(V) (p2)--(V) (p3)--(V);

\draw[med] (q1)--(q2)--(q3)--cycle;
\draw[med] (q1)--(V) (q2)--(V) (q3)--(V);

% optional: hidden/helper edges to mimic book-style construction
\draw[hidden] (q2)--(0.55,0.38,0.0);
\draw[hidden] (q2)--(0.55,0.0,0.78);

% ---------- point + x* ----------
\coordinate (P) at (0.20,0.12,0.55);
\node[dot] at (P) {};
\node[black] at (0.10,-0.03,0.23) {$x^*$};

% mark apex like in the figure
\node[dot] at (V) {};

\end{tikzpicture}

\vspace{1cm}

{\large \today \par}
\vfill

%-----------------------------------------------------------
% Detalle estético inferior
%-----------------------------------------------------------
\rule{0.5\textwidth}{0.5pt}\par
\vspace{0.3cm}
\pagebreak

\end{titlepage}

\pagebreak

\abstract{
\lipsum[]
}


\tableofcontents

\chapter{Introducción}
\section{La Migración}
\section{Casa Monarca}

\chapter{Planteamiento del problema}
\section{El Reto}
\section{Problema a Resolver}
\subsection{Objetivo}

\chapter{Análisis de literatura}


\section{El Problema de la Dieta}
El problema de la dieta es uno de los primeros y más famosos casos de aplicación de la programación lineal. Originado por George Stigler (1945), busca encontrar la combinación más económica de alimentos que satisfaga ciertas necesidades nutricionales. Aunque el modelo original asume divisibilidad total y no considera la palatabilidad, sentó las bases para el modelado nutricional moderno.

Stigler planteó el siguiente problema:

\begin{quote}
  "Para un hombre moderadamente activo,  (economista), pesando 154 libras cuanto de cada una de las 77 comidas deberian ser comsumidas diariamente para que su consumo de 9 nutrientes sea, al menos, igual a la dosis dietetica recomendada (RDA), con el costo de la dieta siendo minima?"\autocite{dantzig1990}
\end{quote}

\begin{table}[H]
\centering
\caption{1943 RDAs for a moderately active 154-pound man.}
\label{tab:rda1943}
\begin{tabular}{@{}l r@{}}
\hline
\textbf{Nutrient} & \textbf{RDA} \\
\hline
Calories & 3,000 kcalories \\
Protein & 70 grams \\
Calcium & 0.8 grams \\
Iron & 12 milligrams \\
Vitamin A & 5,000 IU \\
Thiamine (Vitamin B$_1$) & 1.8 milligrams \\
Riboflavin (Vitamin B$_2$) & 2.7 milligrams \\
Niacin & 18 milligrams \\
Ascorbic Acid (Vitamin C) & 75 milligrams \\
\hline
\end{tabular}
\end{table}


El análisis de este problema es importante, dado a que se da un enfoque en encontrar el la mejor combinación de nutrientes mientras se mantiene el minimo costo posible, lo cual nos da un acercamiento sobre como podemos resolver el problema de satisfacer las necesidades dieteticas de los albergados por la orgranizacion. 

Asimismo, se podría sugerir un enfoque más preciso incluyendo también las necesidades nutricionales para grupos de distintas edades y sexos lo cual aunque puede presentar un problema estocástico, si su modelación matemática es posible podría darnos la oportunidad de tener un modelo más complejo y representativo de la realidad.

El problema sugiere un modelo de la estructura de programaciòn lineal:

\begin{align*}
  \mathbf{Min}\quad x\\
  s.t\quad & d\leq Ax \leq b\\
  x\geq 0
\end{align*}




\section{El Problema de Inventarios}
La teoría de inventarios aborda cómo y cuándo reabastecer productos. En el contexto de alimentos, los modelos deterministas (como el lote económico o EOQ) deben adaptarse para incluir la \textit{perecibilidad}. Nahmias (1982) estableció fundamentos clave sobre cómo modelar productos con vida útil fija, lo cual es esencial para ingredientes como lácteos, carnes y verduras frescas en el albergue.

El problema puede ser modelado usando teorias de control optimo, programación dinamica, y optimización de redes. 

El problema matematico consiste en lo siguiente:

Una tienda tiene en un tiempo $k$ $x_K$ objetos en su inventario. Depues ordena y recibe  $u_k$ objetos y vende $w_k$, donde $w_k$ sigue una distribución de probabilidad. Entonces:

\begin{align}
  x_{x+1}=x_k+u_k-w_k\\
  u_k\geq 0
\end{align}

La tienda quiere minimizar $u_k$, por lo que:

\begin{align}
  \mathbf{Min}\sum_{k=0}^{T}c_k.
\end{align}\autocite{enwiki:1187845455}


\section{El Problema de los Menus}
A diferencia del problema de la dieta pura, el problema de planificación de menús introduce variables binarias e índices de aceptabilidad para garantizar que la comida sea variada y del agrado de los comensales. Balintfy (1975) fue pionero en formular modelos de programación entera para menús institucionales (hospitales, escuelas), asegurando combinaciones gastronómicas lógicas.

Se han desarrollado dos metodos de optimizacion basico para acercarse a este problema 

\end{document}
